\chapter*{Abstract}

Climate change is a problem that humanity needs to combat.
Software engineers can contribute to the fight against global warming by writing energy-efficient software.
This thesis explores which environmental impact the choice of programming language can have.
By writing image processing programs in C++ and Python and benchmarking them using different configurations of software and
hardware, we found that while a programming language in isolation can affect the energy efficiency of a program,
writing software does not happen in isolation.
Therefore, every configuration of software and hardware affects the energy efficiency of a program.
If one part is not optimised, it can increase the total runtime of the program by a significant amount.
Therefore, we can conclude that the choice of programming language can only impact the runtime if all other factors of the program's configuration are optimised.
