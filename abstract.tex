Global warming is a problem that we need to combat. Software engineers can contribute to the fight against global warming by writing energy-efficient software. This thesis explores which environmental impact the choice of programming language can have. We do this by writing an image processing program in C++ and Python and benchmarking them using different configurations of software and hardware. We found that while a programming language in isolation can have an effect on the energy efficiency of a program, writing software does not happen in isolation. This means that every part of the software and hardware configuration influences the energy efficiency of a program. If one part is slightly off, it can increase the total runtime of the program by a significant amount. Therefore, we can conclude that the choice of programming language can only impact the runtime if all other factors of the program's configuration are optimised.
