\chapter{Introduction}
\setcounter{page}{1}

This thesis is about the environmental impact of the choice of programming language. Every operation on a computer requires energy. When extrapolating the usage of a single computer to a global operation, the required energy of single operations becomes significant. Therefore, software engineers need to optimise their programs to require the least amount of energy possible. This thesis researches which part the choice of programming language plays in energy efficiency.
Other research has been performed on the most efficient programming languages, testing several algorithms in different languages and determining the most efficient ones. The research proved that C, Rust and C++ were the most efficient. This thesis experimented with a more realistic program, with disk I/O (disk input and output) and other operations as well as an algorithm. The same program was tested in Python and C++ and the runtime of each operation in the program was benchmarked.
According to a study at Queens University in Ontario, Canada, programmers have limited knowledge about energy efficiency, lack knowledge about the best practice to reduce the energy consumption of software, and are often unsure about how software consumes energy \cite{programmers}.
Therefore, this research could serve as a recommendation on how to reduce energy usage by using the appropriate programming language choice. At the very least, it could help developers think about the environmental impact of their software by putting forth observations on its environmental impact.
This research hypothesises the impact of the choice of programming language on energy efficiency. This hypothesis will investigate through a series of experiments where the same image processing program is benchmarked in C++ and Python in different hardware and software configurations with the aim to observe which language runs fastest.
First, the aim is to give a general introduction to green computing and the used programming languages. A small part of the paper will report previous research on the energy efficiency of programming languages and how this research was utilised in this thesis. Another chapter will illustrate the research hypothesis and explain how it can be proved or disproved. The next chapter will cover the research methods, including the code and profilers. Then, the results of the experiments will be displayed and the results will be discussed in the following chapter. Lastly, there will be a conclusion and bibliography.
