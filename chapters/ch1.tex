\chapter{Introduction}
This thesis is about the environmental impact of the choice of programming language. Every operation on a computer takes up energy. Extrapolating the usage of a single computer to a global operation, the required energy of single operations becomes significant. Therefore, software engineers need to optimise their programs to take up the least amount of energy possible. This thesis researches which part the choice of programming language plays in energy efficiency.
Other research has been done on the most efficient programming languages, testing several algorithms in different languages and determining the most efficient ones, which proved to be C, Rust and C++. We set out to experiment with a more realistic program, with disk I/O (disk input and output) and other operations as well as an algorithm. We tested the same program in Python and C++ and benchmarked the runtime of each operation in the program.
According to a study at Queens University in Ontario, Canada, programmers have limited knowledge about energy efficiency, lack knowledge about the best practice to reduce the energy consumption of software, and are often unsure about how software consumes energy \cite{programmers}.
Therefore, our research is essential to perhaps ultimately serve as a recommendation on how to reduce energy usage by making a programming language choice. At the very least, to make developers think about the environmental impact of their software by putting forth observations on its environmental impact.
This research hypothesises an environmental impact on the choice of programming language. This hypothesis will be proved or disproved through a series of experiments where we benchmark the same image processing program in C++ and Python in different hardware and software configurations. We aim to observe which language runs fastest.
First, we aim to give a general introduction to green computing and the programming languages we will be exploring. A small part of the paper will report previous research on the energy efficiency of programming languages and how we utilised this research in this thesis. Another chapter will illustrate the research hypothesis and explain how it can be proved or disproved. The next chapter will cover the research methods, including the code and profilers. Then we will display the results of the experiments that were done and discuss the results in the following chapter. Lastly, there will be a conclusion and bibliography.
