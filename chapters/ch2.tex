\chapter{Green Computing}
\setcounter{page}{1}

Power electronics are an ever-increasing part of the energy sector. Therefore, a way to conserve energy to aid in the fight against global warming has emerged: green computing.
Green computing is the environmentally responsible and eco-friendly use of computers and their resources \cite{Salama20}. Green computing was first started when the US Environmental Protection Agency launched Energy Star in 1992, a label that stands for energy efficiency in monitors, climate control equipment and other technologies \cite{ENERGYSTAR}.
Later, Swedish TCO development launched the TCO certified program to promote low magnetic and electrical emissions from CRT-based computer displays, energy consumption, ergonomics, and the use of hazardous materials in construction. It is still active, 30 years later, constantly updating its criteria according to technological improvements \cite{TCOCertified}.
There are several aspects to green computing. First of all, the software itself, meaning algorithm optimisation and programming language choice. Choice of deployments can also play a big part. For example, it matters where servers that developers choose to run their code on are located because colder climates and locations close to a water source save energy needed for cooling. Additionally, it is important what kind of energy they use, namely fossil or renewable energy. Another aspect is the make of the servers and physical tools that a developer uses. Some servers are more efficient than others, and some are made with recycled or eco-friendly materials.
The aspect that I would like to focus on in this paper is green software.

\section{Quantifying environmental impact}
One standard to measure environmental impact is by using the Greenhouse Gas Protocol (GHG). The GHG is a “comprehensive global standardised framework to measure and manage greenhouse gas (GHG) emissions from private and public sector operations, value chains and mitigation actions.” \cite{GHGProtocol} Companies like “AMD, Apple, Facebook, Google, Huawei, Intel, and Microsoft publish annual sustainability reports using the GHG Protocol.”\cite[p.855]{9407142}  The GHG categorises emissions into three scopes. “Scope one emissions come from fuel combustion (e.g., diesel, natural gas, and gasoline), refrigerants in offices and data centres, transportation, and the use of chemicals and gases in semiconductor manufacturing.” \cite[p.856]{9407142}. Scope two emissions “come from purchased energy and heat powering semiconductor fabs, offices, and data-centre operation.” \cite[p.856]{9407142} These are important in data centres. Scope three “emissions come from all other activities, including the full upstream and downstream supply chain. They often comprise employee business travel, commuting, logistics, and capital goods”. \cite[p.856]{9407142}
The scope that we can influence with software is scope two. Using AI as an example, a paper by U. Gupta (et al.) called Chasing Carbon: The Elusive Environmental Footprint of Computing describes that “algorithmic optimizations for scale-down systems will drastically cut emissions.” \cite[p.863]{9407142}
We can reason that there is a direct correlation between the speed of a program and the amount of data centres. When a program is less optimised, it runs for a longer period of time, occupying data centres for a longer period of time and therefore, the demand for data centres rises, causing more data centres to be built, which is a CO2 intensive process. According to a paper by Md Abu Bakar Siddik, Arman Shehabi and Landon Marston, “approximately 0.5\% of total US greenhouse gas emissions are attributed to data centres” \cite{Siddik_2021}
