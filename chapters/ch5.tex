\chapter{Research Hypothesis}
This thesis’s research hypothesis is that there is an environmental impact on the choice of programming language for a particular application.
The hypothesis will be proved correct if one programming language performs measurably better than another. This will be measured by benchmarking the same image processing application in two different languages. Observations can be made on the speed of the different parts of the application; the image blurring algorithm, the disk input, the disk output and other operations. We can also measure the overall energy consumption of the program.
Testing the hypothesis will be done according to the scientific method. This means performing an experiment and collecting data in a reproducible manner. The experiment will be controlled, meaning that all variables besides the programming language will be the same as far as that is possible. By this method, we can ensure that the result of the experiment pertains to the programming language and no other factors such as the library, OS, IDE or profiler.
A weakness of our study might be that it is not always possible to eliminate all variables because of compatibility issues with the programming languages. For example, a profiler that works for C++ will not work in the same capacity for Python code and, therefore, might produce slightly different results.
The results of this research could provide guidance to environmentally conscious developers on how to reduce their environmental impact for an image processing project.
