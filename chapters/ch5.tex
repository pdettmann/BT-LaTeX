\chapter{Research Hypothesis}
This thesis’s research hypothesis was that there is an environmental impact based on the choice of programming language for a particular application.
The hypothesis would be proved correct if one programming language performed measurably better than another. This was measured by benchmarking the same image processing application in two languages. Observations were made on the speed of the different parts of the application; the image blurring algorithm, the disk input, the disk output and other operations. The overall energy consumption of the program could also be measured.
Testing the hypothesis was done according to the scientific method. This means performing an experiment and collecting data in a reproducible manner. The experiment was controlled, meaning that all variables besides the programming language were the same, as far as that was possible. By this method, it could be ensured that the result of the experiment pertained to the programming language and no other factors such as the library, OS, IDE or profiler.
A weakness of this study might be that it is not always possible to eliminate all variables because of compatibility issues with the programming languages. For example, a profiler that works for C++ will not work in the same capacity for Python code and, therefore, might produce slightly different results.
The results of this research could guide environmentally conscious developers on how to reduce their environmental impact for an image processing project.
