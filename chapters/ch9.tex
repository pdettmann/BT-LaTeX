\chapter{Conclusion}
This research aimed to identify the environmental impact of the choice of programming language. Based on a series of experiments, it can be concluded that the choice of programming language can only impact the runtime if all other factors of the program's configuration are optimised.
The same program was benchmarked in two different languages and in a multitude of different configurations to find which impact the language would make. It was expected to see C++ perform better than Python for the image processing program. The results did not match these expectations, and experiments were done to explain these results. These experiments resulted in the observation that minor differences in the configuration of a program can make a big difference in total runtime.
A limitation of the research is the scale. Provided with more time and resources, it would be interesting to experiment with the program on more operating systems, different hardware, and more library variations. Additionally, to better understand the pure impact of the programming language, it would be interesting to forgo libraries altogether and write all functionality from scratch to receive ultimate control over the program. Another approach could be to experiment with more programming languages to find the nuances that set them apart for different purposes.
This research contributes the knowledge that when we want to make an environmental impact with our software, we need to optimise every part of the configuration of a program, including choosing the best programming language for the purpose.
